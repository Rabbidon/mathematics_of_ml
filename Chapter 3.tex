\documentclass{article}
\usepackage{enumitem}
\usepackage[utf8]{inputenc}
\usepackage{amsmath}
\usepackage{amssymb}
\usepackage{amsthm}

\newcommand{\chapternumber}{3}
\title{Mathematics for Machine Learning - Chapter \chapternumber\space Solutions}
\author{Edwin Fennell}
\date{}
\newenvironment{QandA}{\begin{enumerate}[label=\chapternumber.\arabic*]\bfseries\boldmath}
	{\end{enumerate}}
\newenvironment{answered}{\par\bigskip\normalfont\unboldmath}{}
\usepackage{lipsum}
\pagestyle{empty}

\begin{document}
	\maketitle
	
	Note - I'm not going to write out the questions here since they are very, very inefficiently posed and no way am I going to TeX all of that.
	
		
	\noindent%
	\begin{QandA}
		\item 
		\begin{answered}
			We can more straightforwardly represent $\langle x,y\rangle$ as 
			\[x^T
			\begin{pmatrix}
				1 & -1 \\
				-1 & 2
			\end{pmatrix}
			y\]
			\begin{itemize}
				\item Bilinearity - we note that tensor multiplication commutes with scalar multiplication and distributes over tensor addition. Therefore our function is linear over both $x$ and $y$.
				\item Symmetry - the output of our function $\langle .,.\rangle$ is a scalar and thus is equal to its own transpose, so $\forall x,y$ we have
				\[
				\langle x,y\rangle
				=
				x^T
				\begin{pmatrix}
					1 & -1 \\
					-1 & 2
				\end{pmatrix}
				y
				=
				\left(
				x^T
				\begin{pmatrix}
					1 & -1 \\
					-1 & 2
				\end{pmatrix}
				y
				\right)^T
				\]
				\[
				=
				y^T
				\begin{pmatrix}
					1 & -1 \\
					-1 & 2
				\end{pmatrix}^T
				x
				=
								y^T
				\begin{pmatrix}
					1 & -1 \\
					-1 & 2
				\end{pmatrix}
				x
				=
				\langle y,x\rangle
				\]
				\item Positive definite - for some reason this chapter doesn't cover diagonalisation so I guess we'll do this manually. 
				
				Consider an arbitrary $v = (v_1,v_2)^T\in\mathbb{R}$. Then
				\[\langle v,v \rangle = 
				v_1^2-2v_1v_2+2v_2^2 = (v_1-v_2)^2 + v_2^2\geq0								
				\]
				with equality iff $v_1=v_2=0$.
			\end{itemize}
		\end{answered}
		
		\item 
		\begin{answered}
			This is not an inner product since the matrix corresponding to this bilinear form is not symmetric. We observe that
			\[
			\langle
			\begin{pmatrix}
				0 \\ 1
			\end{pmatrix},
			\begin{pmatrix}
				1 \\ 0
			\end{pmatrix}
			\rangle = 1
			\]
			but also that
			\[
			\langle
			\begin{pmatrix}
				1 \\ 0
			\end{pmatrix},
			\begin{pmatrix}
				0 \\ 1
			\end{pmatrix}
			\rangle = 0
			\]
			This bilinear form is not symmetric and therefore is not an inner product
		\end{answered}
		
		\item 
		\begin{answered}
			Before we start we should note that
			\[x-y = 
			\begin{pmatrix}
				2 \\ 3 \\ 3
			\end{pmatrix}
			\]
			a. w.r.t. this inner product we have
			\[
			\langle x-y, x-y\rangle
			=
			\begin{pmatrix}
				2 & 3 & 3
			\end{pmatrix}
			\begin{pmatrix}
				2 \\ 3 \\ 3
			\end{pmatrix}
			=
			22
			\]
			and so
			\[||x-y||=\sqrt{22}\]
			b. w.r.t. this inner product we have
			\[
			\langle x-y, x-y\rangle
			=
			\begin{pmatrix}
				2 & 3 & 3
			\end{pmatrix}
			\begin{pmatrix}
				2 & 1 & 0 \\
				1 & 3 & -1 \\
				0 & -1 & 2 \\
			\end{pmatrix}
			\begin{pmatrix}
				2 \\ 3 \\ 3
			\end{pmatrix}
			=
			47
			\]
			and so
			\[||x-y||=\sqrt{47}\]
		\end{answered}
		
		\item
		\begin{answered}
			We define the "angle" $\theta$ w.r.t an inner product $\langle .,. \rangle$ (and its induced norm $||.||$) between two vectors $x,y$ as
			\[\theta = cos^{-1}\left(\frac{\langle x,y \rangle}{||x||,||y||}\right)\]
			where we define $cos^{-1}$ to take values in $[0,\pi]$
			
			a. 
			\[\langle x,y \rangle = x^Ty = -3\]
			\[||x|| = \sqrt{x^Tx} = \sqrt{5}\]
			\[||y|| = \sqrt{y^Ty} = \sqrt{2}\]
			and therefore we have
			\[\theta = cos^{-1}\left(\frac{\langle x,y \rangle}{||x||,||y||}\right) = cos^{-1}\left(\frac{-3}{\sqrt{10}}\right)\approx 161.6^\circ\]
			b.
			\[\langle x,y \rangle = x^TAy = -11\]
			\[||x|| = \sqrt{x^TAx} = \sqrt{18}\]
			\[||y|| = \sqrt{y^TAy} = \sqrt{7}\]
			and therefore we have
			\[\theta = cos^{-1}\left(\frac{\langle x,y \rangle}{||x||,||y||}\right) = cos^{-1}\left(\frac{-11}{\sqrt{126}}\right)\approx 168.5^\circ\]
		\end{answered}
		
		\item
		\begin{answered}
			The way we proceed here is to use the Gram-Schmidt process to create an orthonormal basis for $U$. First we need a basis for $U$, so we had better construct one from our (not necessarily linearly independent) spanning set.
			
			We do this via Gaussian elimination, as before. We start with a matrix where the rows are our spanning set:
			\[
			\begin{pmatrix}
				0 & -1 &  2 &  0 &  2 \\
				1 & -3 &  1 & -1 &  2 \\
				-3 &  4 &  1 &  2 &  1 \\
				-1 & -3 &  5 &  0 &  7 \\
			\end{pmatrix}
			\]
			We switch the first and second rows to get
			\[\begin{pmatrix}
				1 & -3 &  1 & -1 &  2 \\
				0 & -1 &  2 &  0 &  2 \\
				-3 &  4 &  1 &  2 &  1 \\
				-1 & -3 &  5 &  0 &  7 \\
			\end{pmatrix}\]
			We now reduce column 1 using row 1 to get
			\[\begin{pmatrix}
				1 & -3 &  1 & -1 &  2 \\
				0 & -1 &  2 &  0 &  2 \\
				0 &  -5 &  4 &  -1 &  7 \\
				0 & -6 &  6 &  -1 &  9 \\
			\end{pmatrix}\]
			We now reduce column 2 using row 2 to get
			\[\begin{pmatrix}
				1 & -3 &  1 & -1 &  2 \\
				0 & -1 &  2 &  0 &  2 \\
				0 &  0 &  -6 &  -1 &  -3 \\
				0 & 0 &  -6 &  -1 &  -3 \\
			\end{pmatrix}\]
			The last two rows are exact clones so the last row just reduces to 0:
			\[\begin{pmatrix}
				1 & -3 &  1 & -1 &  2 \\
				0 & -1 &  2 &  0 &  2 \\
				0 &  0 &  -6 &  -1 &  -3 \\
				0 &  0 &  0 &  0 &  0 \\
			\end{pmatrix}\]
			Therefore we have
			\[
			\left\{
			\begin{pmatrix}
			1 \\ -3 \\  1 \\ -1 \\  2 \\
			\end{pmatrix}
			,
			\begin{pmatrix}
			0 \\ -1 \\  2 \\  0 \\  2 \\
			\end{pmatrix}
			,
			\begin{pmatrix}
			0 \\  0 \\  -6 \\  -1 \\  -3 \\
			\end{pmatrix}
			\right\}
			\]
			as a basis for $U$. We denote these vectors as $b_1,b_2,b_3$ respectively.
			
			We have a basis, now we can use the Gram-Schmidt process to orthonormalise it.
			Our first non-normalised basis vector is 
			\[u_1 = b_1 = 
			\begin{pmatrix}
				1 \\ -3 \\  1 \\ -1 \\  2 \\
			\end{pmatrix}
			\]
			and so our first normalised basis vector is 
			\[
			e_1 = \frac{u_1}{||u_1||}=
			\frac{1}{4}
			\begin{pmatrix}
				1 \\ -3 \\  1 \\ -1 \\  2 \\
			\end{pmatrix}
			\]
			The next non-normalised basis vector is 
			\[
			u_2 = b_2-\langle e_1,b_2 \rangle e_1 = 
			\begin{pmatrix}
				0 \\ -1 \\  2 \\  0 \\  2 \\
			\end{pmatrix}
			-\frac{9}{16}
			\begin{pmatrix}
				1 \\ -3 \\  1 \\ -1 \\  2 \\
			\end{pmatrix}
			=
			\frac{1}{16}
			\begin{pmatrix}
				-9 \\
				11 \\
				23 \\
				9 \\
				14
			\end{pmatrix}
			\]
			and so our second normalised basis vector is
			\[e_2 = \frac{1}{12\sqrt{7}}
			\begin{pmatrix}
				-9 \\
				11 \\
				23 \\
				9 \\
				14
			\end{pmatrix}
			\]
			The final non-normalised basis vector is 
			\[
			u_3 = b_3 - \langle e_1,b_3 \rangle e_1 - \langle e_2,b_3 \rangle e_2
			\]
			\[
			= 
			\begin{pmatrix}
				0 \\  0 \\  -6 \\  -1 \\  -3 \\
			\end{pmatrix}
			- \frac{11}{16}
			\begin{pmatrix}
				1 \\ -3 \\  1 \\ -1 \\  2 \\
			\end{pmatrix}
			- \frac{3}{16}
			\begin{pmatrix}
				-9 \\
				11 \\
				23 \\
				9 \\
				14
			\end{pmatrix}
			=
			\begin{pmatrix}
				-1 \\
				0 \\
				-1 \\
				0 \\
				1
			\end{pmatrix}
			\]
			and so we have the final normalised basis vector as
			\[e_3 = \frac{1}{\sqrt{3}}
			\begin{pmatrix}
				-1 \\
				0 \\
				-1 \\
				0 \\
				1
			\end{pmatrix}
			\]
			a. Finally, we can now actually attack the question. We can now write the orthogonal projection of $x$ onto $U$ as
			\[\sum_{i=1}^{3}\langle x,e_i \rangle e_i
			=
			\frac{23}{16}
			\begin{pmatrix}
				1 \\ -3 \\  1 \\ -1 \\  2 \\
			\end{pmatrix}
			-
			\frac{1}{16}
			\begin{pmatrix}
				-9 \\
				11 \\
				23 \\
				9 \\
				14
			\end{pmatrix}
			+
			\begin{pmatrix}
				-1 \\
				0 \\
				-1 \\
				0 \\
				1
			\end{pmatrix}
			=
			\begin{pmatrix}
				1 \\ -5 \\ -1 \\ -2 \\ 3
			\end{pmatrix}
			\]
			b. The distance between $x$ and $U$ is equal to the distance between $x$ and its orthogonal projection onto $U$. This distance is therefore
			\[\left\Vert
			\begin{pmatrix}
				-1 \\
				-9 \\
				-1 \\
				4 \\
				1
			\end{pmatrix}
			-
			\begin{pmatrix}
				1 \\ -5 \\ -1 \\ -2 \\ 3
			\end{pmatrix}
			\right\Vert
			=
			\left\Vert
			\begin{pmatrix}
				-2 \\ -4 \\ 0 \\ 6 \\ -2
			\end{pmatrix}
			\right\Vert
			=
			2\sqrt{15}
			\]
		\end{answered}
	\end{QandA}

\end{document}