\documentclass{article}
\usepackage{enumitem}
\usepackage[utf8]{inputenc}
\usepackage{amsmath}
\usepackage{amssymb}
\usepackage{amsthm}

\newcommand{\chapternumber}{2}
\title{Mathematics for Machine Learning - Questions and Solutions}
\author{Edwin Fennell}
\date{}
\newenvironment{QandA}{\begin{enumerate}[label=\chapternumber.\arabic*]\bfseries\boldmath}
	{\end{enumerate}}
\newenvironment{answered}{\par\bigskip\normalfont\unboldmath}{}
\usepackage{lipsum}
\pagestyle{empty}
\begin{document}
	\maketitle
	
	Note - I'm not going to write out the questions here since they are very, very inefficiently posed and no way am I going to TeX all of that.
	
	\noindent%
	\begin{QandA}
		\item
		\begin{answered}
			a. In order to show that this constitutes a group, we need to show four things:
			\begin{itemize}
				\item Closure - if $a,b\in\mathbb{R}$ then clearly $ab+a+b\in\mathbb{R}$.
				Now, suppose that $ab + a + b=-1$. This rearranges to
				\[a(b+1)=-(b+1)\]
				or 
				\[(a+1)(b+1)=0\]
				Therefore if neither $a$ nor $b$ is equal to -1, $a*b$ also cannot be equal to -1, and therefore * is a valid group operation on $\mathbb{R}\backslash\{-1\}$.
				\item Identity - our identity is 0 since for any $a\in\mathbb{R}\backslash\{-1\}$ we have
				\[a*0 = a\cdot0 + 0 + a = a\]
				\item Inverse - given a fixed $a\in\mathbb{R}\backslash\{-1\}$ we want to solve for $x$ in the following:
				\[a*x = ax + x + a = 0\]
				we rearrange to get
				\[x = \frac{-a}{a+1}\]
				Therefore all elements in $\mathbb{R}\backslash\{-1\}$ have inverses under *
				\item Associativity - we consider the respective values of $(a*b)*c$ and $a*(b*c)$ for arbitrary $a,b,c\in\mathbb{R}\backslash\{-1\}$:
				\[(a*b)*c = (a*b)c + a*b + c = abc + ac + bc + ab + a + b + c\]
				\[(a*(b*c) = a(b*c) + b*c + a = abc + ac + bc + ab + a + b + c\]
				and so we have associativity.
						
			\end{itemize}
				
				Now we need to show that the resulting group is Abelian, but this is clear from the definition of * being completely symmetric in its two operands.
				
				\qed
				
				b. Conveniently, from our proof of associativity we know immediately that
				\[3*x*x = 3x^2 + 3x + 3x + x^2 + x + x + 3 = 4x^2 + 8x + 3\]
				Therefore we need to solve $4x^2 + 8x + 3 = 15$, or rather
				\[4x^2 + 8x - 12 = 4(x^2 + 2x -3) = 4(x+3)(x-1)+0\]
				From this we see that the solutions are exactly $x=1,x=-3$
		\end{answered}
		
		\item
		\begin{answered}
			a. We need to show the four group axioms:
			\begin{itemize}
				\item Closure - By definiton of $\oplus$ the result of its application is a congruence class mod $n$. (Well-posedness is another matter but that isn't asked for here).
				\item Identity - the identity is $\bar{0}$ since
				\[\forall a\in\mathbb{Z}, \bar{a} \oplus \bar{0} = \overline{(a + 0)} = \bar{a}\]
				\item Inverses - the inverse of $\bar{a}$ for any $a\in\mathbb{Z}$ is $\overline{-a}$:
				\[\forall a\in\mathbb{Z}, \bar{a} \oplus \overline{-a} = \overline{(a-a)} = \bar{0}\]
				\item Associativity - for any $a,b,c\in\mathbb{Z}$ we have
			\end{itemize}
		\end{answered}
	\end{QandA}
\end{document}