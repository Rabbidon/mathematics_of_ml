\documentclass{article}
\usepackage{enumitem}
\usepackage[utf8]{inputenc}
\usepackage{amsmath}
\usepackage{amssymb}
\usepackage{amsthm}

\newcommand{\chapternumber}{2}
\title{Mathematics for Machine Learning - Questions and Solutions}
\author{Edwin Fennell}
\date{}
\newenvironment{QandA}{\begin{enumerate}[label=\chapternumber.\arabic*]\bfseries\boldmath}
	{\end{enumerate}}
\newenvironment{answered}{\par\bigskip\normalfont\unboldmath}{}
\usepackage{lipsum}
\pagestyle{empty}
\begin{document}
	\maketitle
	
	Note - I'm not going to write out the questions here since they are very, very inefficiently posed and no way am I going to TeX all of that.
	
	\noindent%
	\begin{QandA}
		\item
		\begin{answered}
			a. In order to show that this constitutes a group, we need to show four things:
			\begin{itemize}
				\item Closure - if $a,b\in\mathbb{R}$ then clearly $ab+a+b\in\mathbb{R}$.
				Now, suppose that $ab + a + b=-1$. This rearranges to
				\[a(b+1)=-(b+1)\]
				or 
				\[(a+1)(b+1)=0\]
				Therefore if neither $a$ nor $b$ is equal to -1, $a*b$ also cannot be equal to -1, and therefore * is a valid group operation on $\mathbb{R}\backslash\{-1\}$.
				\item Identity - our identity is 0 since for any $a\in\mathbb{R}\backslash\{-1\}$ we have
				\[a*0 = a\cdot0 + 0 + a = a\]
				\item Inverse - given a fixed $a\in\mathbb{R}\backslash\{-1\}$ we want to solve for $x$ in the following:
				\[a*x = ax + x + a = 0\]
				we rearrange to get
				\[x = \frac{-a}{a+1}\]
				Therefore all elements in $\mathbb{R}\backslash\{-1\}$ have inverses under *
				\item Associativity - we consider the respective values of $(a*b)*c$ and $a*(b*c)$ for arbitrary $a,b,c\in\mathbb{R}\backslash\{-1\}$:
				\[(a*b)*c = (a*b)c + a*b + c = abc + ac + bc + ab + a + b + c\]
				\[(a*(b*c) = a(b*c) + b*c + a = abc + ac + bc + ab + a + b + c\]
				and so we have associativity.
						
			\end{itemize}
				
				Now we need to show that the resulting group is Abelian, but this is clear from the definition of * being completely symmetric in its two operands.
				
				\qed
				
				b. Conveniently, from our proof of associativity we know immediately that
				\[3*x*x = 3x^2 + 3x + 3x + x^2 + x + x + 3 = 4x^2 + 8x + 3\]
				Therefore we need to solve $4x^2 + 8x + 3 = 15$, or rather
				\[4x^2 + 8x - 12 = 4(x^2 + 2x -3) = 4(x+3)(x-1)+0\]
				From this we see that the solutions are exactly $x=1,x=-3$
		\end{answered}
		
		\item
		\begin{answered}
			a. We need to show the four group axioms:
			\begin{itemize}
				\item Closure - By definiton of $\oplus$ the result of its application is a congruence class mod $n$. (Well-posedness is another matter but that isn't asked for here).
				\item Identity - the identity is $\bar{0}$ since
				\[\forall a\in\mathbb{Z}, \bar{a} \oplus \bar{0} = \overline{(a + 0)} = \bar{a}\]
				\item Inverses - the inverse of $\bar{a}$ for any $a\in\mathbb{Z}$ is $\overline{-a}$:
				\[\forall a\in\mathbb{Z}, \bar{a} \oplus \overline{-a} = \overline{(a-a)} = \bar{0}\]
				\item Associativity - we have
				\[\forall a,b,c\in \mathbb{Z}, (\bar{a} \oplus \overline{b})\oplus \overline{c} = \overline{(a+b)}\oplus \overline{c} = \overline{a+b+c}\]
				and also
				\[\forall a,b,c\in \mathbb{Z}, \bar{a} \oplus (\overline{b}\oplus \overline{c}) = \overline{a} \oplus \overline{(b+c)} = \overline{a+b+c}\]
				and so we have associativity. Assuming that the operator $\oplus$ is well-defined, this more or less comes down to "addition is associative".
			\end{itemize}
			Therefore $(\mathbb{Z}_n,\oplus)$ is indeed a group.
			
			b. I'm not going to write out the multiplication table for $\mathbb{Z}_5\backslash\{\overline{0}\}$. I will show that this is a group when I prove the general case in part d of this question. Assuming that it is a group, it is clearly Abelian from the symmetric nature of $\otimes$.
			
			c. Again, I'll use the result from part d. 8 is composite so this is not a group.
			
			d. Suppose that $n$ is composite. Then $\exists$ $a$,$b$ s.t. $1<a,b<n$ and $a\cdot b=n$. Therefore we have 
			\[\overline{a}\otimes\overline{b}=\overline{n}=\overline{0}\]
			Therefore $\mathbb{Z}_n\backslash\{\overline{0}\}$ is not a group since it fails the requirement of closure.
			
			Now, if $n$ is instead prime, then $\mathbb{Z}_n\backslash\{\overline{0}\}$ is a group - we will show this by verifying the group axioms.
			\begin{itemize}
				\item Closure - suppose that $a,b\in\mathbb{Z}\backslash\{\overline{0}\}$. Now, suppose that
				\[ab\equiv 0 \mod{n}\]
				Then $ab=kn$ for some $k\in \mathbb{Z}$. Since $n$ is prime, $a$ and $n$ are coprime, and therefore by Bezout's theorem, $\exists u,v\in\mathbb{Z}$ s.t.
				\[ua+vn=1\]
				Therefore
				\[b = b\cdot 1 = b(ua+vn) = ab\cdot u + bvn = (uk+bv)n\]
				and so we find that $b$ is a multiple of $n$. This is a contradiction since $b\in\mathbb{Z}\backslash\{\overline{0}\}$. Therefore $ab\not\equiv 0 \mod{n}$ and we have 
				\[\overline{a},\overline{b}\neq\overline{0}\implies\overline{ab}\neq\overline{0}\]
				and so we have closure
				\item Identity - the identity is trivially $\overline{1}$
				\item Inverse - for any $\overline{a}\neq\overline{0}$ we have that $a$ and $n$ are coprime. By Bezout's theorem we know that $\exists u,v\in\mathbb{Z}$ s.t.
				\[ua+vn=1\]
				Therefore
				\[\overline{a}\otimes\overline{u}=\overline{au}=\overline{(1-vn)}=\overline{1}\]
				and so we have constructed an inverse for $\overline{a}$
				\item Associativity - exactly the same proof as in part a. Essentially "multiplication is associative".
			\end{itemize}
			Therefore $(\mathbb{Z}_n,\otimes)$ is indeed a group.
		\end{answered}
		
		\item
		\begin{answered}
			Let's check the four group requirements:
			\begin{itemize}
				\item Closure - $\forall x_1,y_1,z_1,x_2,y_2,z_2\in\mathbb{R}$ we have
				\[
				\begin{pmatrix}
					1 & x_1 & z_1 \\ 0 & 1 & y_1 \\ 0 & 0 & 1
				\end{pmatrix}
				\cdot
				\begin{pmatrix}
					1 & x_2 & z_2 \\ 0 & 1 & y_2 \\ 0 & 0 & 1
				\end{pmatrix}
				=
				\begin{pmatrix}
					1 & x_1+x_2 & x_1y_2+z_1+z_2 \\ 0 & 1 & y_1+y_2 \\ 0 & 0 & 1
				\end{pmatrix}
				\]
				so we have closure
				\item Identity - from the above calculation (or simply by knowing what the identity matrix is) we see that 
				\[
				\begin{pmatrix}
					1 & 0 & 0 \\ 0 & 1 & 0 \\ 0 & 0 & 1
				\end{pmatrix}
				\in
				\mathcal{G}\]
				and acts as the identity.
				\item Inverses - we see that $\forall x,y,z\in\mathbb{R}$ we have
				\[
				\begin{pmatrix}
					1 & x_1 & z_1 \\ 0 & 1 & y_1 \\ 0 & 0 & 1
				\end{pmatrix}
				\cdot
				\begin{pmatrix}
					1 & -x_1 & x_1y_1-z_1 \\ 0 & 1 & -y_1 \\ 0 & 0 & 1
				\end{pmatrix}
				=
				\begin{pmatrix}
					1 & 0 & 0 \\ 0 & 1 & 0 \\ 0 & 0 & 1
				\end{pmatrix}
				\]
				\item Associativity - we can just show this manually. Let
				\[x_1,y_1,z_1,x_2,y_2,z_2,x_3,y_3,z_3\in\mathbb{R}\]
				Then 
				\[
				\left(
				\begin{pmatrix}
					1 & x_1 & z_1 \\ 0 & 1 & y_1 \\ 0 & 0 & 1
				\end{pmatrix}
				\cdot
				\begin{pmatrix}
					1 & x_2 & z_2 \\ 0 & 1 & y_2 \\ 0 & 0 & 1
				\end{pmatrix}
				\right)
				\cdot
				\begin{pmatrix}
					1 & x_3 & z_3 \\ 0 & 1 & y_3 \\ 0 & 0 & 1
				\end{pmatrix}
				\]
				\[=\begin{pmatrix}
					1 & x_1+x_2 & x_1y_2+z_1+z_2 \\ 0 & 1 & y_1+y_2 \\ 0 & 0 & 1
				\end{pmatrix}
				\cdot
				\begin{pmatrix}
					1 & x_3 & z_3 \\ 0 & 1 & y_3 \\ 0 & 0 & 1
				\end{pmatrix}
				\]
				\[=
				\begin{pmatrix}
					1 & x_1+x_2+x_3 & x_1y_2 + x_1y_3 + x_2y_3 + z_1+z_2+z_3 \\ 0 & 1 & y_1+y_2+y_3 \\ 0 & 0 & 1
				\end{pmatrix}\]
				and
				\[
				\begin{pmatrix}
					1 & x_1 & z_1 \\ 0 & 1 & y_1 \\ 0 & 0 & 1
				\end{pmatrix}
				\cdot
				\left(
				\begin{pmatrix}
					1 & x_2 & z_2 \\ 0 & 1 & y_2 \\ 0 & 0 & 1
				\end{pmatrix}
				\cdot
				\begin{pmatrix}
					1 & x_3 & z_3 \\ 0 & 1 & y_3 \\ 0 & 0 & 1
				\end{pmatrix}
				\right)
				\]
				\[=				\begin{pmatrix}
					1 & x_1 & z_1 \\ 0 & 1 & y_1 \\ 0 & 0 & 1
				\end{pmatrix}
				\cdot
				\begin{pmatrix}
					1 & x_2 + x_3 & x_2y_3 + z_2 + z_3 \\ 0 & 1 & y_2 + y_3 \\ 0 & 0 & 1
				\end{pmatrix}
				\]
				\[=
				\begin{pmatrix}
					1 & x_1+x_2+x_3 & x_1y_2 + x_1y_3 + x_2y_3 + z_1+z_2+z_3 \\ 0 & 1 & y_1+y_2+y_3 \\ 0 & 0 & 1
				\end{pmatrix}\]
				and so we have associativity.
			\end{itemize}
			Therefore $(\mathcal{G},\cdot)$ is a group. It is not Abelian though. We can see this from observing that
			\[
			\begin{pmatrix}
				1 & 1 & 1 \\ 0 & 1 & 1 \\ 0 & 0 & 1
			\end{pmatrix}
			\cdot
			\begin{pmatrix}
				1 & 1 & 1 \\ 0 & 1 & 0 \\ 0 & 0 & 1
			\end{pmatrix}
			=
			\begin{pmatrix}
				1 & 2 & 2 \\ 0 & 1 & 1 \\ 0 & 0 & 1
			\end{pmatrix}
			\]
			and
			\[
			\begin{pmatrix}
				1 & 1 & 1 \\ 0 & 1 & 0 \\ 0 & 0 & 1
			\end{pmatrix}
			\cdot
			\begin{pmatrix}
				1 & 1 & 1 \\ 0 & 1 & 1 \\ 0 & 0 & 1
			\end{pmatrix}
			=
			\begin{pmatrix}
				1 & 2 & 3 \\ 0 & 1 & 1 \\ 0 & 0 & 1
			\end{pmatrix}
			\]
		\end{answered}
		
		\item
		\begin{answered}
			The first product is not computable since the column count of the first matrix is not equal to the row count of the second. All the other products are valid. I will not compute them here.
		\end{answered}
	\end{QandA}
\end{document}